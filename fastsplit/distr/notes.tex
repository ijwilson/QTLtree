\documentclass{article}

\begin{document}

A bit of text about the organisation of this code.

I am keeping all the working code for the GAW data in
IHG/fastsplit/src, and copying the library files into this
distribution as needed.  I do not include any of the old QTLTree
program files so these can be edited locally.  The only problem this
may present is that tags files will not work well, and I have to
remember not to edit anything but program files locally.


The code that does this stuff is all in liubraries anyway so things
are pretty easy.

The big problems are example files -- I would like to have them but
they would be very large, so the best bet may be to simulate a data
set using sima or something else and then get the data.
 

Any files for the \textbf{new} code are kept locally, in QTLTree/src

any files for interface with R code is in QTLTree/R


I don't think that I have permission to work with the GAW data - but I
should enable the code to work with the raw GAW data if anyone has it.


\section{Program Notes}
QTLtree plan

   1. Print out code
   2. Input/output formats
   3. Redo simple programs to make use of already written libraries
   4. do I need program for Case/Control and QTLs
   5. utility programs
   6. R library instead
TYhe program is easy - some stuff about this is on google docs - copy it over when I can.

read options

read data

do stuff

write output



\end{document}
